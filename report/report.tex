\documentclass[a4paper]{article}

\usepackage[utf8x]{inputenc}
\usepackage[danish]{babel}
\usepackage{amsmath}
\usepackage{amssymb}
\usepackage{listings}

\parindent 0pt
\parskip 5pt

\newtheorem{theorem}{Sætning}

\lstset{language=Python}

\title{DiS2 projekt 2015\\Primtalstest, moduloregning i store beregninger}
\author{Jonas Aaslet, Kasper Brandt og Truls Asheim}

\begin{document}

\begin{titlepage}
\maketitle
\end{titlepage}


\section{Introduktion}

Denne opgave gennemgår flere forskellige metoder til bestemmelse af
store primtal med den egenskab at enten $n-1$ eller $n+1$ er trivielt
faktoriserbare. Denne egenskab kan vi udnytte til at lave en række
tests som beviser hvorvidt et tal er et primtal. 

Vi har implementeret flere forskellige primtalstests i Python 


\section{Lucas primtals test}
Lucas primtalstesten tilhører gruppen af $n-1$ primtaltests ud fra
tallet $n-1$ kan bestemme om $n$ er primtal.

\begin{theorem}[Lucas primtalstest]
  Lad $n > 0$ og $q$ være alle primtalsfaktorer af $n - 1$. Hvis der 
  findes et heltal $1 < a < n$ så

\begin{equation}
a^{n-1}\equiv 1 \quad (\text{mod}\,n)
\end{equation}
og
\begin{equation}
a^{(n-1)/q} \not\equiv 1\quad (\text{mod}\,n)
\end{equation}
er $n$ et primtal.

\end{theorem}

Metodens resultat er afhængig af valget af $a$. Hvis 

For at sætningen kan bruges i en praktisk primtalstest kræves det at
alle vi kender alle primtalsfaktorer for $n - 1$. Da det er
tidsskrevende at finde alle faktorer for et stort tal er vi i praksis
interesseret i metoder som kan bevise om et tal er et primtal ved kun
at kende enkelte af dets primtalsfaktorer. En bedre 
o
\section{Pocklington primality test}
Pocklington testen er en forbedret version af Lucas testen, som ikke
kræver at man kender alle primtalsfaktoriseringerne.

Hvis man kender en række primtalsfaktoriseringer, q af $n-1$ hvor F er
faktoriseringerne mulitipliseret, $n-1=FR$ og $F>R$, kan man bruge
Pocklington testen til at finde ud af om n er et primtal.


\begin{theorem}[Pocklington test]

Hvis der for hver primtalsfaktor $q$ af $F$ findes et heltal $a > 1$ således at

\begin{equation}
a^{n-1}=1(\mod n)
\end{equation}
og
\begin{equation}
\gcd(a^{(n-1)/q}-1 = 1
\end{equation}
for hver q, så er n et primtal.
\end{theorem}

\subsection{Eksempel}
Vi har $n=71$ og de kendte faktorer af 70 er \{10\}.

Vi udvælger tilfældigt $a=48$.

\begin{align}
48^{70/10}-1 &= 587068342271\\
\gcd(587068342271,71) &= 71
\end{align}


Efter som resultatet er forskelligt fra 1, prøver vi et nyt tilfældigt a.

\begin{align}
a &= 43\\
43^{70/10}-1 &= 271818611106 \\
\gcd(271818611106,71) &= 1 \\
\end{align}

Her er resultatet 1, og derfor ved vi at n er et primtal.

\subsection{Implemetering}
\lstinputlisting{../pocklington.py}

Funktionen tager følgende input:
\begin{itemize}
	\item n: Det primtal vi vil teste.
	\item factors: Listen af kendte primtalsfaktoriseringer af n-1.
	\item acc: Hvor mange tilfældige værdier af a, koden skal teste.
\end{itemize}

Vi starter med at finde F. Derefter tjekker vi om $F<\sqrt{n-1}$. Hvis
den er, så har vi ikke nok primtalsfaktoriseringer.

Vi kører et loop igennem acc gange, hvor vi i hver af dem finder et
tilfældigt a mellem 2 og n-1. Vi tjekker for alle
primtalsfaktoriseringer om Pocklingtons test virker. Hvis den gør,
returnere vi true.


\section{Lucas-Lehmer test}




Lucas-Lehmer testen kan kan bestemme om et primtal $n$, for hvilket det
gælder at $n + 1$ kan faktoriseres trivielt, rent faktisk er et
primtal. 


Vi starter med at definere funktionerne $U$ og $V$ som er rekursivt
defineret som 

\begin{align*}
U(0) = 0,\;U(1) = 1
\end{align*}


\begin{theorem}

\begin{equation}
U(n+1) = 0 (\mod n)
\end{equation}
og
\begin{equation}
U((n+1)/r) \neq 0 (\mod n)
\end{equation}

\end{theorem}

Et sertilfælde af denne sætning er Lucas-Lehmer testen. Denne
fremkommer når man definerer $S(k) = V(2^{k+1}/2^{2^k}$ og kan bruges
til at teste Mersenne tals primalitet.

\begin{theorem}{Lucas-Lehmer Test}
Mersenne-tallet $M(n) = 2^n -1$ er et primtal hvis og kun hvis $S(n -
2) = 0 (mod M)$ hvor $S(k)$ er følgen $S(0) = 4$ og $S(k + 1) = S(k)^2
- 2$.
\end{theorem}

\paragraph{Eksempel}

b

%%% Local Variables:
%%% mode: latex
%%% TeX-master: "report"
'%%% End:

\section{Kombinerede tests}
Den kombineret sætning bruger de andre sætninger, til at lave en
metode som kun skal bruge faktoriseringer $n-1$ og $n+1$, som ganget
sammen giver kubikroden af n. Dette forbedrer redu

Vi har et primtal $n$, og nogle primtalsfaktoriseringer af $n-1$ og $n+1$
som ganget samm=en giver $F_1$ og $F_2$.

Et krav er at $n-1=F_1R_1$, $n+1=F_2R_2$, $gcd(F_1,R_1)=1$ og $gcd(F_2,R_2)=1$

Hvis Pocklington gælder for $n-1$ og $F_1$ og //todo: snak om U-funktionen
så ved vi at følgende gælder:

Hvis $n<\max(F_1^2F_2/2,F_1F_2^2/2)$ så er n et primtal.


\subsection{Implemetering}
\lstinputlisting{../combined.py}






%%% Local Variables:
%%% mode: latex
%%% TeX-master: "report"
%%% End:

\end{document}




