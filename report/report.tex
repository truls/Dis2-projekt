\documentclass[a4paper]{article}

\usepackage[utf8x]{inputenc}
\usepackage[danish]{babel}
\usepackage{listings}

\newtheorem{theorem}{Sætning}

\title{DiS2 projekt 2015\\Primtalstest, moduloregning i store beregninger}
\author{Jonas Aaslet, Kasper Brandt og Truls Asheim}

\newcommand{\mod}{\text{mod}}

\begin{document}

\begin{titlepage}
\maketitle
\end{titlepage}



\section{Introduktion}
Primtalsbeviser [

\section{Lucas primtals test}
Lucas primtalstesten tilhører gruppen af $n-1$ primtaltests ud fra
tallet $n-1$ kan bestemme om $n$ er primtal.

\begin{theorem}[Lucas primtalstest]
  Lad $n > 0$ og $q$ være alle primtalsfaktorer af $n - 1$. Hvis der
  findes et heltal $1 < a < n$ så

\begin{equation}
a^{n-1}\equiv 1 \quad (\text{mod}\,n)
\end{equation}
og
\begin{equation}
a^{(n-1)/q} \nequiv 1\quad (\text{mod}\,n)
\end{equation}
er $n$ et primtal.

\end{theorem}

Det er vigtig at bemærke at denne test kræver at vi kender alle
primtalsfaktorer for $n-1$. Da 



\end{document}




