\documentclass[a4paper]{article}

\usepackage[utf8x]{inputenc}
\usepackage[danish]{babel}
\usepackage{amsmath}
\usepackage{amssymb}
\usepackage{listings}
\usepackage{amsfonts}
\usepackage{amsthm,thmtools}

\parindent 0pt
\parskip 5pt

\PrerenderUnicode{æ}
\PrerenderUnicode{ø}
\PrerenderUnicode{å}

\newcommand{\Uline}[1]{\underline{\underline{#1}}}
\newcommand{\nats}{\mathbb{N}}
\newcommand{\ints}{\mathbb{Z}}
\newcommand{\reals}{\mathbb{R}}
\newcommand{\complexs}{\mathbb{C}}
\newcommand{\complexes}{\mathbb{C}}
\newcommand{\polys}{\mathfrak{P}}
\newcommand{\imp}{\Rightarrow}
\newcommand{\bimp}{\Leftrightarrow}
\newcommand{\andL}{\wedge}
\newcommand{\orL}{\vee}
\newcommand{\logfrac}[2]{\frac{#1'(#2)}{#1(#2)}}
\newcommand{\logfracz}[1]{\logfrac{#1}{z}}
\newcommand{\logffracz}{\logfracz{f}}
\newcommand{\loghfracz}{\logfracz{h}}
\newcommand{\itau}{2\pi i}
\newcommand{\invitau}{\frac{1}{2\pi i}}
\newcommand{\pg}[1]{\left(#1\right)}
\newcommand{\ip}[1]{\left<#1\right>}
\newcommand{\onehalf}{\frac{1}{2}}

\newcommand{\abs}[1]{\left|#1\right|}

\newcommand{\mmgrp}[1]{(\ints/#1\ints)^*}

\newtheorem{theorem}{Sætning}

\lstset{language=Python}

\title{DiS2 projekt 2015\\Primtalstest, moduloregning i store beregninger}
\author{Jonas Aaslet, Kasper Brandt og Truls Asheim}

\begin{document}

\begin{titlepage}
\maketitle
\end{titlepage}


\section{Introduktion}

Denne opgave gennemgår flere forskellige metoder til bestemmelse af
store primtal med den egenskab at enten $n-1$ eller $n+1$ er trivielt
faktoriserbare. Denne egenskab kan vi udnytte til at lave en række
tests som beviser hvorvidt et tal er et primtal. 

Vi har implementeret flere forskellige primtalstests i Python 


\section{Lucas primtals test}
Lucas primtalstesten tilhører gruppen af $n-1$ primtaltests ud fra
tallet $n-1$ kan bestemme om $n$ er primtal.

\begin{theorem}[Lucas primtalstest]
  Lad $n > 0$ og $q$ være alle primtalsfaktorer af $n - 1$. Hvis der 
  findes et heltal $1 < a < n$ så

\begin{equation}
a^{n-1}\equiv 1 \quad (\text{mod}\,n)
\end{equation}
og
\begin{equation}
a^{(n-1)/q} \not\equiv 1\quad (\text{mod}\,n)
\end{equation}
er $n$ et primtal.

\end{theorem}

Metodens resultat er afhængig af valget af $a$. Hvis 

For at sætningen kan bruges i en praktisk primtalstest kræves det at
alle vi kender alle primtalsfaktorer for $n - 1$. Da det er
tidsskrevende at finde alle faktorer for et stort tal er vi i praksis
interesseret i metoder som kan bevise om et tal er et primtal ved kun
at kende enkelte af dets primtalsfaktorer. En bedre 
o
\section{Pocklington primality test}
Pocklington testen er en forbedret version af Lucas testen, som ikke
kræver at man kender alle primtalsfaktoriseringerne.

Hvis man kender en række primtalsfaktoriseringer, q af $n-1$ hvor F er
faktoriseringerne mulitipliseret, $n-1=FR$ og $F>R$, kan man bruge
Pocklington testen til at finde ud af om n er et primtal.


\begin{theorem}[Pocklington test]

Hvis der for hver primtalsfaktor $q$ af $F$ findes et heltal $a > 1$ således at

\begin{equation}
a^{n-1}=1(\mod n)
\end{equation}
og
\begin{equation}
\gcd(a^{(n-1)/q}-1 = 1
\end{equation}
for hver q, så er n et primtal.
\end{theorem}

\subsection{Eksempel}
Vi har $n=71$ og de kendte faktorer af 70 er \{10\}.

Vi udvælger tilfældigt $a=48$.

\begin{align}
48^{70/10}-1 &= 587068342271\\
\gcd(587068342271,71) &= 71
\end{align}


Efter som resultatet er forskelligt fra 1, prøver vi et nyt tilfældigt a.

\begin{align}
a &= 43\\
43^{70/10}-1 &= 271818611106 \\
\gcd(271818611106,71) &= 1 \\
\end{align}

Her er resultatet 1, og derfor ved vi at n er et primtal.

\subsection{Implemetering}
\lstinputlisting{../pocklington.py}

Funktionen tager følgende input:
\begin{itemize}
	\item n: Det primtal vi vil teste.
	\item factors: Listen af kendte primtalsfaktoriseringer af n-1.
	\item acc: Hvor mange tilfældige værdier af a, koden skal teste.
\end{itemize}

Vi starter med at finde F. Derefter tjekker vi om $F<\sqrt{n-1}$. Hvis
den er, så har vi ikke nok primtalsfaktoriseringer.

Vi kører et loop igennem acc gange, hvor vi i hver af dem finder et
tilfældigt a mellem 2 og n-1. Vi tjekker for alle
primtalsfaktoriseringer om Pocklingtons test virker. Hvis den gør,
returnere vi true.


\section{Lucas-Lehmer test}




Lucas-Lehmer testen kan kan bestemme om et primtal $n$, for hvilket det
gælder at $n + 1$ kan faktoriseres trivielt, rent faktisk er et
primtal. 


Vi starter med at definere funktionerne $U$ og $V$ som er rekursivt
defineret som 

\begin{align*}
U(0) = 0,\;U(1) = 1
\end{align*}


\begin{theorem}

\begin{equation}
U(n+1) = 0 (\mod n)
\end{equation}
og
\begin{equation}
U((n+1)/r) \neq 0 (\mod n)
\end{equation}

\end{theorem}

Et sertilfælde af denne sætning er Lucas-Lehmer testen. Denne
fremkommer når man definerer $S(k) = V(2^{k+1}/2^{2^k}$ og kan bruges
til at teste Mersenne tals primalitet.

\begin{theorem}{Lucas-Lehmer Test}
Mersenne-tallet $M(n) = 2^n -1$ er et primtal hvis og kun hvis $S(n -
2) = 0 (mod M)$ hvor $S(k)$ er følgen $S(0) = 4$ og $S(k + 1) = S(k)^2
- 2$.
\end{theorem}

\paragraph{Eksempel}

b

%%% Local Variables:
%%% mode: latex
%%% TeX-master: "report"
'%%% End:

\section{Kombinerede tests}
Den kombineret sætning bruger de andre sætninger, til at lave en
metode som kun skal bruge faktoriseringer $n-1$ og $n+1$, som ganget
sammen giver kubikroden af n. Dette forbedrer redu

Vi har et primtal $n$, og nogle primtalsfaktoriseringer af $n-1$ og $n+1$
som ganget samm=en giver $F_1$ og $F_2$.

Et krav er at $n-1=F_1R_1$, $n+1=F_2R_2$, $gcd(F_1,R_1)=1$ og $gcd(F_2,R_2)=1$

Hvis Pocklington gælder for $n-1$ og $F_1$ og //todo: snak om U-funktionen
så ved vi at følgende gælder:

Hvis $n<\max(F_1^2F_2/2,F_1F_2^2/2)$ så er n et primtal.


\subsection{Implemetering}
\lstinputlisting{../combined.py}






%%% Local Variables:
%%% mode: latex
%%% TeX-master: "report"
%%% End:

\subsection{Lucas test}
\begin{theorem}
Lad $n > 1$. $n$ er da et primtal hvis og kun hvis der for hver primfaktor $q$ af $n-1$ findes et
heltal $1<a<n$ så
\begin{enumerate}
	\item $a^{n-1} \equiv 1 \pmod{n}$
	\item $a^{\frac{(n-1)}{q}} \not \equiv 1 \pmod{n}$
\end{enumerate}
\begin{proof}
	Lad os starte med at bemærke at det første krav implicerer at $a$ og $n$ er primiske.
	Hvorfor det? Fordi vi har $a^{n-1}>n$ da $a$ enten skulle være $1$ eller $a^{n-1}$ kun
	ville være kongruent med $a^{n-1}$ hvis $a^{n-1} < n$, eller kongruent med $0$ hvis
	$a^{n-1} = n$. Deraf ser vi at Euklids algoritme vil give os $\gcd(a^{n-1},n)=1$.
	Da $a$ jo deler sine primfaktorer kan $a$ og $n$ heller ikke have nogen faktorer tilfældes,
	hvorfor de må være primiske.

	Gruppen $\mmgrp{n}$ kan højst have orden $n-1$ (der er $n-1$ naturlige tal mindre end $n$).

	Fra det første krav har vi at $n-1$ må være et multiplum af ordenen af $a$.
	Hvis ordenen af $a$ er mindre end $n-1$ og vi betegner ordenen af $a$ ved $m$ så
	må der være et $r$ så $mr=n-1$. Hvis vi lader $q_0$ være en af primfaktorene i $r$, og
	$s = \frac{r}{q_0}$ så er $msq_0=n-1$ og $a^{ms} \equiv 1 \pmod{n}$, så den anden betingelse
	ville altså ikke være opfyldt hvis ordenen af $a$ var mindre end $n-1$. Deraf kan
	vi altså slutte at ordenen af $a$ må være $n-1$, og derfor også ordenen af
	$\mmgrp{n}$, hvormed $n$ må være et primtal.

	Omvendt hvis $n$ er et primtal så findes der et element der er generator for
	$\mmgrp{n}$, som da må have elementorden $n-1$.
\end{proof}
\end{theorem}

Hvis $n$ er sammensat så kalder vi et $a$ hvor det første punkt ikke gælder for et Fermat
vidne for at $n$ er sammensat, og et $a$ hvor det første punkt gælder for en Fermat løgner.

Hvis $n$ er sammensat kan vi ikke med sikkerhed vise det uden at prøve alle $n-1$ muligheder
igennem. Men hvis den første betingelse ikke er opfyldt så ved vi fra Fermats lille sætning
at $n$ må være sammensat. Heldigvis er sandsynligheden for at få et $a$ der er et Fermat vidne
over 50\%.

Vi kan komme ud for at der ikke er nogen Fermat vidner for et givent $n$ selvom det
er et sammensat tal -- i det tilfælde er $n$ et Carmichael tal. Heldigvis er Carmichael tal
sjældne, der er f.eks. kun 246683 Carmichael tal op til $10^{16}$. Vi ved at antallet af
Carmichael tal under $x$ vokser langsommere and $kx$ -- mere specifikt viste Erdős en øvre grænse
på
$$
C(x)<xe^{\frac{-k\log x \log\log\log x}{\log\log x}},
$$
for en eller anden (positiv) konstant $k$. Det afgørende her er at sandsynligheden for at få
et Carmichael tal bliver negligibel i forhold til sandsynligheden for at få et sammensat tal
når tallene bliver store nok, så vi kan nøjes med at se på sandsynligheden for at få et
Fermat vidne for ikke-Carmichael sammensatte tal.

Lad os nu antage at $n$ er et sammensat ikke-Carmichael tal, og lad $a$ være et Fermat vidne.
Lad nu $a_1,a_2,\ldots,a_l$ være Fermat løgnere. Da er for alle $i$
$$
	(aa_i)^{n-1} \equiv a^{n-1}a_i^{n-1}\equiv a^{n-1} \not \equiv 1,
$$
så alle $aa_i$ for $i=1,2,\ldots,l$ er Fermat vidner.
\subsection{Pocklingtons test}
\begin{theorem}[Pockingtons sætning]
Lad $n-1=q^kR$ hvor $q$ er et primtal som ikke deler $R$. Hvis der er et heltal 
\end{theorem}

\end{document}




