
\section{Lucas-Lehmer test}




Lucas-Lehmer testen kan kan bestemme om et primtal $n$, for hvilket det
gælder at $n + 1$ kan faktoriseres trivielt, rent faktisk er et
primtal. 


Vi starter med at definere funktionerne $U$ og $V$ som er rekursivt
defineret som 

\begin{align*}
U(0) = 0,\;U(1) = 1
\end{align*}


\begin{theorem}

\begin{equation}
U(n+1) = 0 (\mod n)
\end{equation}
og
\begin{equation}
U((n+1)/r) \neq 0 (\mod n)
\end{equation}

\end{theorem}

Et sertilfælde af denne sætning er Lucas-Lehmer testen. Denne
fremkommer når man definerer $S(k) = V(2^{k+1}/2^{2^k}$ og kan bruges
til at teste Mersenne tals primalitet.

\begin{theorem}{Lucas-Lehmer Test}
Mersenne-tallet $M(n) = 2^n -1$ er et primtal hvis og kun hvis $S(n -
2) = 0 (mod M)$ hvor $S(k)$ er følgen $S(0) = 4$ og $S(k + 1) = S(k)^2
- 2$.
\end{theorem}

\paragraph{Eksempel}

b

%%% Local Variables:
%%% mode: latex
%%% TeX-master: "report"
'%%% End:
