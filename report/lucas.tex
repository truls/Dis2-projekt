
\section{Lucas primtals test}
Lucas primtalstesten tilhører gruppen af $n-1$ primtaltests ud fra
tallet $n-1$ kan bestemme om $n$ er primtal.

\begin{theorem}[Lucas primtalstest]
  Lad $n > 0$ og $q$ være alle primtalsfaktorer af $n - 1$. Hvis der 
  findes et heltal $1 < a < n$ så

\begin{equation}
a^{n-1}\equiv 1 \quad (\text{mod}\,n)
\end{equation}
og
\begin{equation}
a^{(n-1)/q} \not\equiv 1\quad (\text{mod}\,n)
\end{equation}
er $n$ et primtal.

\end{theorem}

Sætningens første betingelse stammer fra fermats lille sætning og
giver et definitivt svar på om $n$ er et sammensat tal, men
sandsynliggør blot at $n$ er et primtal.

Metodens resultat er afhængig af valget af $a$. Hvis 

For at sætningen kan bruges i en praktisk primtalstest kræves det at
alle vi kender alle primtalsfaktorer for $n - 1$. Da det er
tidsskrevende at finde alle faktorer for et stort tal er vi i praksis
interesseret i metoder som kan bevise om et tal er et primtal ved kun
at kende enkelte af dets primtalsfaktorer. En 

\subsection{Implementering}

\lstinputlisting{../lucas.py}


%%% Local Variables:
%%% mode: latex
%%% TeX-master: "report"
%%% ispell-local-dictionary: "danish"
%%% End:
