\subsection{Lucas test}
\begin{theorem}
Lad $n > 1$. $n$ er da et primtal hvis og kun hvis der for hver primfaktor $q$ af $n-1$ findes et
heltal $1<a<n$ så
\begin{enumerate}
	\item $a^{n-1} \equiv 1 \pmod{n}$
	\item $a^{\frac{(n-1)}{q}} \not \equiv 1 \pmod{n}$
\end{enumerate}
\begin{proof}
	Lad os starte med at bemærke at det første krav implicerer at $a$ og $n$ er primiske.
	Hvorfor det? Fordi vi har $a^{n-1}>n$ da $a$ enten skulle være $1$ eller $a^{n-1}$ kun
	ville være kongruent med $a^{n-1}$ hvis $a^{n-1} < n$, eller kongruent med $0$ hvis
	$a^{n-1} = n$. Deraf ser vi at Euklids algoritme vil give os $\gcd(a^{n-1},n)=1$.
	Da $a$ jo deler sine primfaktorer kan $a$ og $n$ heller ikke have nogen faktorer tilfældes,
	hvorfor de må være primiske.

	Gruppen $\mmgrp{n}$ kan højst have orden $n-1$ (der er $n-1$ naturlige tal mindre end $n$).

	Fra det første krav har vi at $n-1$ må være et multiplum af ordenen af $a$.
	Hvis ordenen af $a$ er mindre end $n-1$ og vi betegner ordenen af $a$ ved $m$ så
	må der være et $r$ så $mr=n-1$. Hvis vi lader $q_0$ være en af primfaktorene i $r$, og
	$s = \frac{r}{q_0}$ så er $msq_0=n-1$ og $a^{ms} \equiv 1 \pmod{n}$, så den anden betingelse
	ville altså ikke være opfyldt hvis ordenen af $a$ var mindre end $n-1$. Deraf kan
	vi altså slutte at ordenen af $a$ må være $n-1$, og derfor også ordenen af
	$\mmgrp{n}$, hvormed $n$ må være et primtal.

	Omvendt hvis $n$ er et primtal så findes der et element der er generator for
	$\mmgrp{n}$, som da må have elementorden $n-1$.
\end{proof}
\end{theorem}

Hvis $n$ er sammensat så kalder vi et $a$ hvor det første punkt ikke gælder for et Fermat
vidne for at $n$ er sammensat, og et $a$ hvor det første punkt gælder for en Fermat løgner.

Hvis $n$ er sammensat kan vi ikke med sikkerhed vise det uden at prøve alle $n-1$ muligheder
igennem. Men hvis den første betingelse ikke er opfyldt så ved vi fra Fermats lille sætning
at $n$ må være sammensat. Heldigvis er sandsynligheden for at få et $a$ der er et Fermat vidne
over 50\%.

Vi kan komme ud for at der ikke er nogen Fermat vidner for et givent $n$ selvom det
er et sammensat tal -- i det tilfælde er $n$ et Carmichael tal. Heldigvis er Carmichael tal
sjældne, der er f.eks. kun 246683 Carmichael tal op til $10^{16}$. Vi ved at antallet af
Carmichael tal under $x$ vokser langsommere and $kx$ -- mere specifikt viste Erdős en øvre grænse
på
$$
C(x)<xe^{\frac{-k\log x \log\log\log x}{\log\log x}},
$$
for en eller anden (positiv) konstant $k$. Det afgørende her er at sandsynligheden for at få
et Carmichael tal bliver negligibel i forhold til sandsynligheden for at få et sammensat tal
når tallene bliver store nok, så vi kan nøjes med at se på sandsynligheden for at få et
Fermat vidne for ikke-Carmichael sammensatte tal.

Lad os nu antage at $n$ er et sammensat ikke-Carmichael tal, og lad $a$ være et Fermat vidne.
Lad nu $a_1,a_2,\ldots,a_l$ være Fermat løgnere. Da er for alle $i$
$$
	(aa_i)^{n-1} \equiv a^{n-1}a_i^{n-1}\equiv a^{n-1} \not \equiv 1,
$$
så alle $aa_i$ for $i=1,2,\ldots,l$ er Fermat vidner.
\subsection{Pocklingtons test}
\begin{theorem}[Pockingtons sætning]
Lad $n-1=q^kR$ hvor $q$ er et primtal som ikke deler $R$. Hvis der er et heltal 
\end{theorem}
