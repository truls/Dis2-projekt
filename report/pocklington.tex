\section{Pocklington primality test}
Pocklington testen er en forbedret version af Lucas testen, som ikke kræver at man kender alle primtalsfaktoriseringerne.

Hvis man kender en række primtalsfaktoriseringer, q af $n-1$ hvor F er faktoriseringerne mulitipliseret, $n-1=FR$ og $F>R$, kan man bruge Pocklington testen
til at finde ud af om n er et primtal.

Hvis $a^{n-1}=1(mod n)$ og $gcd(a^{(n-1)/q}-1=1$ for hver q, så er n et primtal.

\subsection{Eksempel}
Vi har $n=71$ og de kendte faktorer af 70 er \{10\}.

Vi udvælger tilfældigt $a=48$.

$48^{70/10}-1=587068342271$

$gcd(587068342271,71)=71$

Efter som resultatet er forskelligt fra 1, prøver vi et nyt tilfældigt a.

$a=43$

$43^{70/10}-1=271818611106$

$gcd(271818611106,71)=1$

Her er resultatet 1, og derfor ved vi at n er et primtal.

\subsection{Implemetering}
//TODO: Lav om til pæn kode
def pocklington(n,factors,acc):
	F=1
	for f in factors:
		F*=f
	if F<math.sqrt(n-1):
		print "Need more factors"
		return
	for i in range(acc):
		a=random.randint(2,n-1)
		if pow(a,n-1, n) != 1:
			return False
		prime=True
		for q in factors:
			if gcd(a**((n-1)/q)-1,n)!=1:
				prime=False
				break
		if prime:
			return True
	return False

Funktionen tager følgende input:
\begin{itemizer}
	\item n: Det primtal vi vil teste.
	\item factors: Listen af kendte primtalsfaktoriseringer af n-1.
	\item acc: Hvor mange tilfældige værdier af a, koden skal teste.
\end{itemizer}

Vi starter med at finde F. Derefter tjekker vi om $F<\sqrt{n-1}$. Hvis den er, så har vi ikke nok primtalsfaktoriseringer.

Vi kører et loop igennem acc gange, hvor vi i hver af dem finder et tilfældigt a mellem 2 og n-1. Vi tjekker for alle primtalsfaktoriseringer om Pocklingtons test virker. Hvis den gør, returnere vi true.
