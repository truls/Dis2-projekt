\section{Pocklington primality test}
Pocklington testen er en forbedret version af Lucas testen, som ikke
kræver at man kender alle primtalsfaktoriseringerne.

Hvis man kender en række primtalsfaktoriseringer, q af $n-1$ hvor F er
faktoriseringerne mulitipliseret, $n-1=FR$ og $F>R$, kan man bruge
Pocklington testen til at finde ud af om n er et primtal.


\begin{theorem}[Pocklington test]

Hvis der for hver primtalsfaktor $q$ af $F$ findes et heltal $a > 1$ således at

\begin{equation}
a^{n-1}=1(\mod n)
\end{equation}
og
\begin{equation}
\gcd(a^{(n-1)/q}-1 = 1
\end{equation}
for hver q, så er n et primtal.
\end{theorem}

\subsection{Eksempel}
Vi har $n=71$ og de kendte faktorer af 70 er \{10\}.

Vi udvælger tilfældigt $a=48$.

\begin{align}
48^{70/10}-1 &= 587068342271\\
\gcd(587068342271,71) &= 71
\end{align}


Efter som resultatet er forskelligt fra 1, prøver vi et nyt tilfældigt a.

\begin{align}
a &= 43\\
43^{70/10}-1 &= 271818611106 \\
\gcd(271818611106,71) &= 1 \\
\end{align}

Her er resultatet 1, og derfor ved vi at n er et primtal.

\subsection{Implemetering}
\lstinputlisting{../pocklington.py}

Funktionen tager følgende input:
\begin{itemize}
	\item n: Det primtal vi vil teste.
	\item factors: Listen af kendte primtalsfaktoriseringer af n-1.
	\item acc: Hvor mange tilfældige værdier af a, koden skal teste.
\end{itemize}

Vi starter med at finde F. Derefter tjekker vi om $F<\sqrt{n-1}$. Hvis
den er, så har vi ikke nok primtalsfaktoriseringer.

Vi kører et loop igennem acc gange, hvor vi i hver af dem finder et
tilfældigt a mellem 2 og n-1. Vi tjekker for alle
primtalsfaktoriseringer om Pocklingtons test virker. Hvis den gør,
returnere vi true.
